\documentclass[10pt,letterpaper]{article}

% Copyright (c) 2012 Cies Breijs
%
% The MIT License
%
% Permission is hereby granted, free of charge, to any person obtaining a copy
% of this software and associated documentation files (the "Software"), to deal
% in the Software without restriction, including without limitation the rights
% to use, copy, modify, merge, publish, distribute, sublicense, and/or sell
% copies of the Software, and to permit persons to whom the Software is
% furnished to do so, subject to the following conditions:
%
% The above copyright notice and this permission notice shall be included in
% all copies or substantial portions of the Software.
%
% THE SOFTWARE IS PROVIDED "AS IS", WITHOUT WARRANTY OF ANY KIND, EXPRESS OR
% IMPLIED, INCLUDING BUT NOT LIMITED TO THE WARRANTIES OF MERCHANTABILITY,
% FITNESS FOR A PARTICULAR PURPOSE AND NONINFRINGEMENT. IN NO EVENT SHALL THE
% AUTHORS OR COPYRIGHT HOLDERS BE LIABLE FOR ANY CLAIM, DAMAGES OR OTHER
% LIABILITY, WHETHER IN AN ACTION OF CONTRACT, TORT OR OTHERWISE, ARISING FROM,
% OUT OF OR IN CONNECTION WITH THE SOFTWARE OR THE USE OR OTHER DEALINGS IN THE
% SOFTWARE.

%%% LOAD AND SETUP PACKAGES

\usepackage[margin=0.75in]{geometry} % Adjusts the margins

\usepackage{multicol} % Required for multiple columns of text

\usepackage{mdwlist} % Required to fine tune lists with a inline headings and indented content

\usepackage{relsize} % Required for the \textscale command for custom small caps text

\usepackage{hyperref} % Required for customizing links
\usepackage{xcolor} % Required for specifying custom colors
\definecolor{dark-blue}{rgb}{0.15,0.15,0.45} % Defines the dark blue color used for links
\hypersetup{colorlinks,linkcolor={dark-blue},citecolor={dark-blue},urlcolor={dark-blue}} % Assigns the dark blue color to all links in the template

\usepackage{tgtermes} % Use the TeX Gyre Pagella font throughout the document
\usepackage[T1]{fontenc}
\usepackage{microtype} % Slightly tweaks character and word spacings for better typography

\pagestyle{empty} % Stop page numbering

%----------------------------------------------------------------------------------------
%	DEFINE STRUCTURAL COMMANDS
%----------------------------------------------------------------------------------------

\newenvironment{indentsection} % Defines the indentsection environment which indents text in sections titles
{\begin{list}{}{\setlength{\leftmargin}{\newparindent}\setlength{\parsep}{0pt}\setlength{\parskip}{0pt}\setlength{\itemsep}{0pt}\setlength{\topsep}{0pt}}}{\end{list}}

\newcommand*\maintitle[2]{\noindent{\LARGE \textbf{#1}}\ \ \ \emph{#2}\vspace{0.3em}} % Main title (name) with date of birth or subtitle

\newcommand*\roottitle[1]{\subsection*{#1}\vspace{-0.3em}\nopagebreak[4]} % Top level sections in the template

\newcommand{\headedsection}[3]{\nopagebreak[4]\begin{indentsection}\item[]\textscale{1.1}{#1}\hfill#2#3\end{indentsection}\nopagebreak[4]} % Section title used for a new employer

\newcommand{\headedsubsection}[3]{\nopagebreak[4]\begin{indentsection}\item[]\textbf{#1}\hfill\emph{#2}#3\end{indentsection}\nopagebreak[4]} % Section title used for a new position

\newcommand{\bodytext}[1]{\nopagebreak[4]\begin{indentsection}\item[]#1\end{indentsection}\pagebreak[2]} % Body text (indented)

\newcommand{\inlineheadsection}[2]{\begin{basedescript}{\setlength{\leftmargin}{\doubleparindent}}\item[\hspace{\newparindent}\textbf{#1}]#2\end{basedescript}\vspace{-1.7em}} % Section title where body text starts immediately after the title

\newcommand*\acr[1]{\textscale{.85}{#1}} % Custom acronyms command

\newcommand*\bull{\ \ \raisebox{-0.365em}[-1em][-1em]{\textscale{4}{$\cdot$}} \ } % Custom bullet point for separating content

\newlength{\newparindent} % It seems not to work when simply using \parindent...
\addtolength{\newparindent}{\parindent}

\newlength{\doubleparindent} % A double \parindent...
\addtolength{\doubleparindent}{\parindent}

\newcommand{\breakvspace}[1]{\pagebreak[2]\vspace{#1}\pagebreak[2]} % A custom vspace command with custom before and after spacing lengths
\newcommand{\nobreakvspace}[1]{\nopagebreak[4]\vspace{#1}\nopagebreak[4]} % A custom vspace command with custom before and after spacing lengths that do not break the page

\newcommand{\spacedhrule}[2]{\breakvspace{#1}\hrule\nobreakvspace{#2}} % Defines a horizontal line with some vertical space before and after it

\hyphenation{Some-long-word} 

\begin{document}
\maintitle{Kamal Pangeni, Ph.D.}{}

\noindent\textsmaller{+}1 (801) 921-2420\bull
\href{mailto:kamalpangeni@gmail.com}{kamalpangeni@gmail.com}\bull
\href{https://github.com/kamalpangeni}
{github.com/kamalpangeni}\\
\href{https://www.linkedin.com/in/kamal-pangeni-31242a29/}
{www.linkedin.com/in/kamalpangeni}\bull
\href{https://scholar.google.com/citations?user=MJPS73gAAAAJ&hl=en}
{scholar.google.com/\textsmaller{+}KamalPangeniPhD}

\spacedhrule{0.9em}{-0.4em}

\roottitle{Professional Profile}
Trained as a theoretical high energy physicist, I enjoy solving complex problems with innovative ideas. My background in physics provides a solid foundation for quantitative problem solving skills. I have extensive experience in data manipulation, data analysis, building machine learning models, and deploying the product to end users at a large organization.\\
\hrule
%%%%%%%%%%%%%%%%%%%
%Areas of Expertiese
%%%%%%%%%%%%%%%%%%%%
\roottitle{Areas of Expertise} 
 $\bullet$ Supervised and Unsupervised Machine Learning \hspace{2.6cm}
 $\bullet$ Deep Learning\\
 $\bullet$ Natural Language Processing\hspace{5.4cm}
 $\bullet$ Statistical analysis\\
 $\bullet$ Mathematical modeling \hspace{6.1cm}
 $\bullet$ Algorithm development and implementation\\
 $\bullet$ Monte Carlo modeling and simulations \hspace{3.95cm}
 $\bullet$ Problem solving
 
\hrule
\roottitle{Skills}
\textbf{Computer Languages}: Python, Mathematica, SQL, MATLAB, C++, java, \LaTeX \\
\textbf{Data analysis and visualization}: Numpy, Pandas, Matplotlib, Seaborn, QlikView, QlikSense\\
\textbf{ Machine Learning}: Scikit-Learn, Keras, XGboost, LightGBM, Tensorflow, Pytorch\\
\textbf{ AutoML}: H20.ai, DataRobot\\
\textbf{Languages}: English, Nepali, Hindi, Urdu\\
\hrule
%%%%%%%%%%%%%%%%%%%%%%%%%%%%%%%%%%%%%%%%%%%%%%%%%%%%%%%%%%%%%%%%%%%%%%%
% Data Science 
%%%%%%%%%%%%%%%%%%%%%%%%%%%%%%%%%%%%%%%%%%%%%%%%%%%%%%%%%%%%%%%%%%%%%%%
%\hrule

\roottitle{ Experience}

\headedsubsection
{ Data Scientist -  Machine Learning \hspace{3ex} @Mercy Hospital, St. Louis}
{ Feb 2018-Present}
{\bodytext{ I develop machine learning models to drive operational efficiency and create profitability.
\begin{itemize}
\item  Built and deployed machine learning models to predict the Diagnosis Related Group (DRG) for inpatient admissions. Within a year of its deployment, it has already saved more than 7 million dollars in hospital revenue. This project was selected as a finalist (in cost and efficiency category) at Mercy wide innovation conference. 
\item Developed Qlik Sense app to share the predictions from the model with medical coders who are currently using it to audit and review medical coding at Mercy.
\item Built a Natural Language Processing model based on deep learning to predict the diagnosis for inpatient admission from physician notes.
\item Estimated the financial impact of elective surgery cancellations due to COVID19.
\item Used Natural Language Processing techniques to automate the cancer registry case findings.
%\item Won first ever Mercy hackathon \href {https://www.linkedin.com/feed/update/urn:li:activity:6582628165401792512/}{[LINK]}. 
\end{itemize}
}}
\headedsubsection
{ Graduate Research Assistant \hspace{9ex}@Washington University, St. Louis}
{ Aug 2013-Aug 2017}
{\bodytext{  I worked on several aspects of quantum chromodynamics (QCD) at nonzero temperature and density. My principle achievements included:
\begin{itemize}
	\item Development and implementation of algorithms to find the saddle point of a multi-dimensional function, and perform multi-dimensional integration numerically,  as required to study the phase structure of QCD and liquid-gas system.
	\item Computation of the mass spectrum and correlation functions in lattice models of QCD by numerically diagonalizing matrices of large dimensions.
	\item Perform multi-dimensional integral numerically using Monte Carlo techniques to calculate the rate of neutrino emission from neutron stars. 
	\end{itemize} }}
\headedsubsection
{ Undergraduate Research Assistant \hspace{3ex} @Brigham Young University}
{ May 2008-Aug 2011}
{\bodytext{ $\bullet$ I successfully constructed and tested a time-of-flight ion spectrometer that was later used in single photon radiation experiment. As part of the testing of the apparatus, I  collected a large set of data from experiments and performed data analysis (primarily Mathematica).\\
		$\bullet$ In making a brief transition to algebraic topology, I quickly learned new ideas and applied it to my research. My main focus was to identify different triangulations and circle packing on the surface of a sphere and use it to investigate the properties of knots with less than nine twist regions.}
}

%%%%%%%%%%%%%%%%%%%%%%%%%%%%%%%%%%%%%%%%%%%%%%%%%%%%%%%%%%%%%%%%%%%%%%%
% Experience
%%%%%%%%%%%%%%%%%%%%%%%%%%%%%%%%%%%%%%%%%%%%%%%%%%%%%%%%%%%%%%%%%%%%%%%
\newpage
\maintitle{Kamal Pangeni, Ph.D.}{}

\noindent\textsmaller{+}1 (801) 921-2420\bull
\href{mailto:kamalpangeni@wustl.edu}{kamalpangeni@wustl.edu}\bull
\href{https://github.com/kamalpangeni}
{github.com/kamalpangeni}\\
\href{https://www.linkedin.com/in/kamal-pangeni-31242a29/}
{www.linkedin.com/in/kamalpangeni}\bull
\href{https://scholar.google.com/citations?user=MJPS73gAAAAJ&hl=en}
{scholar.google.com/\textsmaller{+}KamalPangeniPhD}
\hrule
%%%%%%%%%%%%%%%%%%%
%Education
%%%%%%%%%%%%%%%%%%%%
\roottitle{Education}

\headedsection
{Washington University}
{\textsc{St. Louis MO}} {

\headedsubsection
{Doctor of Philosophy in Physics 3.89/4}{\textsc{Aug 2017}}
{}
{\bodytext{Dissertation:
    {\textit{Topics in QCD at nonzero temperature and density}}}}

\headedsubsection
{Masters of Science in Physics 3.79/4}{\textsc{May 2013}}
{}{}}


\headedsection
{Brigham Young University}
{\textsc{Provo,UT}} {

\headedsubsection
{Bachelors of Science in Physics and Mathematics 3.7/4}{\textsc{Jan 2007-Aug 2011}}
{}
{\bodytext{}}
}

\hrule

\roottitle{Machine Learning Courses on Coursera}
\headedsubsection
{\href{https://www.coursera.org/account/accomplishments/specialization/86SNN7EEXUL2}{Reinforcement Learning Specialization (4 courses)}}{March 2020}{}
\headedsubsection
%{\href{https://www.coursera.org/account/accomplishments/certificate/Q495SV77QKBY}{Sample-based Learning Methods}}{Jan 2020}{}
%\headedsubsection
%{\href{https://www.coursera.org/account/accomplishments/certificate/GXSM8ZMNXLYR}{Fundamentals of Reinforcement Learning}}{Dec 2019}{}
%\headedsubsection
{\href{https://www.coursera.org/account/accomplishments/certificate/KFLEJGRZMYQZ}{Introduction to TensorFlow for AI, ML and DL}}{June 2019}{}
\headedsubsection
{\href{https://www.coursera.org/account/accomplishments/specialization/P8F8U69WUCVD}{Deep Learning Specialization (5 Courses)}}{May 2019} {}
\headedsubsection
{\href{https://www.coursera.org/account/accomplishments/verify/WAQ4WS7BCBDS}{How to Win a Data Science Competition}}{June 2018}{}
\headedsubsection
{\href{https://www.coursera.org/account/accomplishments/certificate/R3Q26XAHTCHD}{Machine Learning}}{Nov 2017}{}

%%%%%%%%%%%%%%%%%%%%
%Awards
%%%%%%%%%%%%%%%%%%%%%%
\hrule
\roottitle{ Honors and Awards} 
$\bullet$ {\bf Finalist in Mercy wide innovation conference}\hfill{\bf Oct 2019}\\
$\bullet$ {\bf Winner of 1st Mercy Hackathon} [\href {https://www.linkedin.com/feed/update/urn:li:activity:6582628165401792512/}{link}]\hfill{\bf Sept 2019}\\
 $\bullet$ {\bf Arthur L. Hughes Fellow}, Washington University[\href{https://physics.wustl.edu/graduate#secondary}{link}]\hfill {\bf Summer 2012}\\
 $\bullet$ {\bf University Fellow}, Washington University [\href{https://physics.wustl.edu/graduate#secondary}{link}] \hfill {\bf 2011 \& 2013 }\\
 $\bullet$ {\bf Harvery Fletcher Scholarship}, Brigham Young University\hfill {\bf 2010/2011}\\
 $\bullet$ {\bf Academic Scholarship}, Brigham Young University\hfill{\bf 2007-2010}\\
 $\bullet$ {\bf Physics Department Scholarship}, Brigham Young University \hfill{\bf 2007-2010}\\

%%%%%%%%%%%%%%%%%%%%%%%%%%%%%%%%%%%%%%%%%%%%%%%%%%%%%%%%%%%%%%%%%%%%%%%
%Publications
%%%%%%%%%%%%%%%%%%%%%%%%%%%%%%%%%%%%%%%%%%%%%%%%%%%%%%%%%%%%%%%%%%%%%%%
\hrule
\roottitle{Publications}
(Note: In high energy physics, name of authors are listed in alphabetical order)\\

\begin{itemize}
\item  M. Alford, \textbf{K. Pangeni} and A. Windisch ``Color Superconductivity and Charge Neutrality in Yukawa Theory", Physics Review Letters 120, 082701 (2018)
 \item H. Nishimura, M. Ogilvie and \textbf{K. Pangeni}, ``Liquid-Gas Phase Transition and $CK$ Symmetry in Quantum Field Theories'', Physical Review D 95, 076003(2017).
\item  Mark Alford and \textbf{Kamal Pangeni} ``Gap bridging enhancement of modified Urca process in nuclear matter.'', Physical Review C 95, 015802 (2017).
 \item H. Nishimura, M. Ogilvie and \textbf{K. Pangeni}, ``Complex mass spectrum in lattice QCD with static quarks at strong coupling'', Physical Review D 93, 094501 (2016).
 \item  H. Nishimura, M. Ogilvie and \textbf{K. Pangeni}, ``Complex saddle points and disorder lines in QCD at finite temperature and density'', Physical Review D 91,054004 (2015). 
 \item H. Nishimura, M. Ogilvie and \textbf{K. Pangeni}, ``Complex saddle points in QCD at finite temperature and density'', Physical Review D 90,045039 (2014). 
 \end{itemize}
\end{document}



